\documentclass{emulateapj}
%\documentclass[12pt,preprint]{aastex}

\usepackage{graphicx}
\usepackage{float}
\usepackage{amsmath}
\usepackage{epsfig,floatflt}


\begin{document}

\title{A spectacular title}

\author{Stig-Nicolai Foyn, Christer Dreierstad}

\email{stignicf@student.matnat.uio.no, chrisdre@student.matnat.uio.no}

\altaffiltext{1}{Institute of Physics, University of
  Oslo, P.O.\ Box 1029 Blindern, N-0315 Oslo, Norway}


\begin{abstract}
\end{abstract}
\keywords{cosmic microwave background --- cosmology: observations --- methods: statistical}

\section{Introduction}
\label{sec:introduction}



\section{Method}
\label{sec:method}
\subsection{Buckling beam}
Considering a beam that is fastened in both ends which moves up and down when a force F is exerted on it can be described by a differential equation
%
\begin{gather}
    \gamma \frac{d²u(x)}{dx²} = -Fu(x),
\end{gather}
%
where $u(x)$ is the displacement of the beam in the y-direction and $\gamma$ is a constant that is defined by the properties of the beam, such as the rigidity. Since the beam is fastened in both ends, the displacement $u(x) = 0$ and the sum outer forces on the beam is zero. 


\section{Data}
\label{sec:data}


\section{Results}
\label{sec:results}



\section{Conclusions}
\label{sec:conclusions}





%\begin{figure}[t]
%
%\mbox{\epsfig{figure=filename.eps,width=\linewidth,clip=}}
%
%\caption{Description of figure -- explain all elements, but do not
%draw conclusions here.}
%\label{fig:figure_label}
%\end{figure}



\begin{deluxetable}{lccc}
%\tablewidth{0pt}
\tablecaption{\label{tab:results}}
\tablecomments{Summary of main results.}
\tablecolumns{4}
\tablehead{Column 1  & Column 2 & Column 3 & Column 4}
\startdata
Item 1 & Item 2 & Item 3 & Item 4
\enddata
\end{deluxetable}



\begin{acknowledgements}
\end{acknowledgements}


\begin{thebibliography}{}

\bibitem[]{}

\end{thebibliography}


\end{document}
