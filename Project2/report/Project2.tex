\documentclass{emulateapj}
%\documentclass[12pt,preprint]{aastex}

\usepackage{graphicx}
\usepackage{float}
\usepackage{amsmath}
\usepackage{epsfig,floatflt}
\setcitestyle{square}

\DeclareMathAlphabet{\mathcal}{OMS}{cmsy}{m}{n}
\SetMathAlphabet{\mathcal}{bold}{OMS}{cmsy}{b}{n}


\begin{document}

\title{Project 2}

\author{Stig-Nicolai Foyn, Christer Dreierstad}

\email{stignicf@student.matnat.uio.no, chrisdre@student.matnat.uio.no}

\altaffiltext{1}{Institute of Physics, University of
  Oslo, P.O.\ Box 1029 Blindern, N-0315 Oslo, Norway}


\begin{abstract}
\end{abstract}
\keywords{buckling beam --- quantum dots --- harmonic oscillator- and Coulomb potentials --- methods: scaling of differential equations, Jacobi's method}

\section{Introduction}
\label{sec:introduction}



\section{Method}
\label{sec:method}
The method presented will show how to solve an eigenvalue problem using Jacobi's method. The first problem is a classical one involving a beam fastened in both ends. Further the method is transformed to a quantum mechanical three dimensional quantum dot problem, starting with one electron and moving over to two electrons.

\subsection{Jacobi's method for solving an eigenvalue problem}
Solving the eigenvalue problem with Jacobi's method means solving an equation on the form
%
\begin{gather}\label{eq:B=SAS}
    \bold{S}^T\bold{A}\bold{S} = \bold{B},
\end{gather}
%
where $\bold{S}$ is an orthogonal matrix, so it has the property $\bold{S}^{-1} = \bold{S}^T$. The resulting matrix $\bold{B}$ will consist of the eigenvalues along the diagonal. The matrix $\bold{S}$ is such that the elements along the diagonal are 1, and zero elsewhere, except four elements which are such that their position are decided by the largest element of the off-diagonal of $\bold{A}$. The largest element of $\bold{A}$ is found at a position $(k,l)$, and these indices are used to define the position of the mentioned four elements of $\bold{S}$, which are such that $s_{kk} = s_{ll} = \cos\theta$ and $s_{kl} = -s_{lk} = -\sin\theta$. This is considered as a rotation of an angle $\theta$ in the n-dimensional Euclidean space. When solving \eqref{eq:B=SAS} the resulting equations to solve are
%
\begin{align*}
    &b_{ii} = a_{ii}, i \neq k, i \neq l \\
    &b_{ik} = a_{ik}c - a_{il}s, i\neq k, i\neq l \\
    &b_{il} = a_{il}c + a_{ik}s, i \neq k, i\neq l \\
    &b_{kk} = a_{kk} c^2 - 2a_{kl}cs+ a_{ll}s^2 \\
    &b_{ll} = a_{ll} c^2 + 2a_{kl}cs a_{kk}s^2 \\
    &b_{kl} = \left(a_{kk} - a_{ll}\right)cs + a_{kl}\left(c^2 - s^2\right),
\end{align*}
where $c = \cos\theta$ and $s = \sin\theta$. The main goal is to reduce the off-diagonal elements of $\bold{A}$ to zero in such a way that $\bold{B}$ has the eigenvalues along the diagonal, the reduction of the elements is done by choosing the angle $\theta$. We make a requirement that the off-diagonal elements of $\bold{B}$ is zero;
%
\begin{align*}
    b_{kl} &= 0 = \left(a_{kk} - a_{ll}\right)cs + a_{kl} \left(c^2 - s^2\right),
\end{align*}
%
which can then be written as
%
\begin{gather*}
    \frac{\left(a_{ll} - a_{kk}\right)}{a_{kl}}cs = \left(c^2 - s^2\right).
\end{gather*}
%
Defining $\tau = \left(a_{ll} - a_{kk}\right)/2a_{kl}$ and $\tan\theta = t = s/c$, and multiplying the equation above by $1/c^2$ we obtain a quadratic equation 
%
\begin{gather*}
    t^2 + 2\tau t - 1 = 0,
\end{gather*}
%
which is used to define the best fit $\theta$ to reduce the off-diagonal elements of $\bold{A}$. The solution to the quadratic equation is $t = -\tau \pm \sqrt{1 + \tau^2}$. If $\tau$ becomes large, the numerical accuracy will drop and $t \approx -\tau \pm \tau$. By the following transformation
%
\begin{align*}
    t_{\pm} &= -\tau \pm \sqrt{1+\tau^2}\left(\frac{\tau \pm \sqrt{1+\tau^2}}{\tau \pm \sqrt{1+\tau^2}}\right) \\
    &= \frac{-\tau^2 + \left(1 + \tau^2\right)}{\tau \pm \sqrt{1 + \tau^2}} = \frac{1}{\tau \pm \sqrt{1+\tau^2}}.
\end{align*}
%
and by splitting the equation such that when $\tau < 0$ we choose the negative sign in front of the square root the aforementioned issue is avoided. The final equations for t are
%
\begin{align*}
    t_+ = \frac{1}{\tau + \sqrt{1 + \tau^2}}, \tau \geq 0 \\
    t_- = \frac{1}{\tau - \sqrt{1 + \tau^2}}, \tau < 0
\end{align*}
%
From the definition of $t = \tan\theta$, s can be expressed as
%
\begin{align*}
    s = \sin\theta = \tan\theta \cos\theta = tc,
\end{align*}
and c can be expressed as
\begin{align*}
    c &= \cos\theta = \frac{\cos\theta}{\sqrt{\cos^2\theta + \sin^2\theta}} = \frac{1}{\sqrt{1 + \tan^2\theta}}  = \frac{1}{\sqrt{1 + t^2}}.
\end{align*}

 
\subsubsection{Numerical implementation}
When implementing Jacobi's method numerically, $\bold{B}$ is not explicitly created, instead the elements of $\bold{A}$ are changed during each iteration. This allows the program to run with large matrices (large n), without taking up more memory thasn needed. This means the left hand side of the six previous equations are changed so that they are valid for the elements of $\bold{A}$ instead of $\bold{B}$, meaning $b_{ii} \rightarrow a_{ii}$, $b_{ik} \rightarrow a_{ik}$ and so forth.

///EIGENVECTORS////

\subsection{Buckling beam}
Considering a beam that is fastened in both ends which moves up and down when a force F is exerted on it can be described by a differential equation
%
\begin{gather}\label{eq:du2/dx2}
    R \frac{d^2 u(x)}{dx^2} = -Fu(x),
\end{gather}
%
where $u(x)$ is the displacement of the beam in the y-direction and $R$ is a constant that is defined by the properties of the beam, such as the rigidity. Since the beam, with length L, is fastened in both ends, the displacement at the end points $u(0) = u(L) = 0$, so we have Dirichlet boundary condition as in the previous paper \cite{1}, and the sum of outer forces exerted on the beam is zero. Introducing the dimensionless variable $\rho = x/L \Rightarrow x = \rho L$, where $\rho \in [0,1]$. Introducing this into Eq. \eqref{eq:du2/dx2} gives
%
\begin{gather*}
    \frac{R}{L}\frac{d^2 u(\rho)}{d\rho^2} = - FLu(\rho)
    \Rightarrow \frac{d^2 u(\rho)}{d\rho^2} = -\frac{FL^2}{R}u(\rho).
\end{gather*}
%
Substituting $\lambda = FL^2/R$ transforms the problem to an eigenvalue problem, with discretized $\lambda$'s as eigenvalues with corresponding eigenvectors. The methods presented in the previous paper regarding the solution of a differential equation will be applied to this problem. The left hand size of the equation to solve,
%
\begin{gather}\label{eq:du/dx=lambda}
    \frac{d^2 u(\rho)}{dx^2} = -\lambda u(\rho)
\end{gather}
%
can be approximated as
%
\begin{gather*}
    \frac{d^2 u(\rho)}{dx^2} = \frac{u(\rho + h) - 2u(\rho) + u(\rho - h)}{h^2} +  \mathcal{O}(h^2),
\end{gather*}
%
where the step size $h = \left(\rho_{max} - \rho_{min}\right)/n$, which is the total length of the beam divided by the resolution of the calculations n. The maximum and minimum value of the dimensionless length variable $\rho$ is at the end points of the beam, such that $\rho_{max} = 1$ and $\rho_{min} = 0$. Following the method from the previous paper the discretized equation to solve is
%
\begin{gather*}
    -\frac{u_{i+1} - 2u_i + u_{i-1}}{h^2} = \lambda u_i.
\end{gather*}
Which by the method of the previous paper can be transferred to a matrix eigenvalue problem;
%
\begin{equation*}
    \begin{bmatrix} d& a & 0   & 0    & \dots  &0     & 0 \\
                                a & d & a & 0    & \dots  &0     &0 \\
                                0   & a & d & a  &0       &\dots & 0\\
                                \dots  & \dots & \dots & \dots  &\dots      &\dots & \dots\\
                                0   & \dots & \dots & \dots  &a  &d & a\\
                                0   & \dots & \dots & \dots  &\dots       &a & d\end{bmatrix} 
                                 \begin{bmatrix} u_1 \\ u_2 \\ u_3 \\ \dots \\ u_{N-2} \\ u_{N-1}\end{bmatrix} = \lambda \begin{bmatrix} u_1 \\ u_2 \\ u_3 \\ \dots \\ u_{N-2} \\ u_{N-1}\end{bmatrix} . 
\label{eq:matrixse} 
\end{equation*}
%
Where a and b are defined as $a = -1/h^2$ and $b = 2/h^2$. For this specific eigenvalue problem the analytical eigenvalues are
%
\begin{gather}\label{eq:eigenvals_analytical}
    \lambda_j = d + 2a\cos\left(\frac{j\pi}{n+1}\right)
\end{gather}
for $j = 1,2,...,n-1$. 

\subsection{Quantum dots, one electron}
The method derived in the section for the buckling beam is general for a problem that can be described by a differential equation on the form of \eqref{eq:du/dx=lambda}. The method can be extended to a quantum mechanical problem, if the problem can be transformed into a differential equation as mentioned. The radial part of the Schr$\ddot{\mathrm{o}}$dinger equation with the orbital angular momentum number $l = 0$ is
%
\begin{equation*}
    -\frac{\hbar^2}{2m}\left(\frac{1}{r^2}\frac{d}{dr}r^2\frac{d}{dr}\right)R(r) + V(r)R(r) = ER(r),
\end{equation*}
%
where $r \in [0,\infty)$, the potential $V(r)$ is the harmonic oscillator potential $V(r) = (1/2)kr^2$, where $k = mw^2$ and $\omega$ is the oscillator frequency. The allowed energies are 
%
\begin{equation*}
    E_{n0} = \left(2n + \frac{3}{2}\right),
\end{equation*}
%
for  n = 0, 1, 2,.... Substituting $R(r) = u(r)/r$:
%
\begin{equation*}
    -\frac{\hbar^2}{2m}\frac{1}{r^2}\frac{d}{dr}r^2 \frac{d}{dr}\frac{u(r)}{r} + V(r)\frac{u(r)}{r} = E\frac{u(r)}{r}.
\end{equation*}
%
Evaluating the derivatives in the above equation:
%
\begin{align*}
    \frac{1}{r^2}\frac{d}{dr}r^2\frac{d}{dr}\frac{u(r)}{r} &=  \frac{1}{r^2}\frac{d}{dr}r^2\left(\frac{\frac{du}{dr}r - u}{r^2}\right) \\
    &= \frac{1}{r^2}\frac{d}{dr}\left(\frac{du}{dr}r - u\right) \\
    &= \frac{1}{r^2}\left(\frac{d^2u}{dr^2}r + \frac{du}{dr} - \frac{du}{dr}\right) \\
    &= \frac{1}{r}\frac{d^2u}{dr^2},
\end{align*}
%
Inserting the above and multiplying by r yields
%
\begin{equation*}
    -\frac{\hbar^2}{2m}\frac{d^2u}{dr^2} + V(r)u(r) = Eu(r).
\end{equation*}
%
where the boundary conditions of $u$ are $u(0) = u(\infty) = 0$. Defining a dimensionless variable $\rho = r/\alpha$ transforms the potential to $V(\rho) = (1/2)k\alpha^2\rho^2$, and the Schr$\ddot{\mathrm{o}}$dinger's equation becomes
%
\begin{equation*}
    -\frac{\hbar}{2m\alpha^2}\frac{d^2u(\rho)}{d\rho^2} + \frac{1}{2}k\alpha^2\rho^2u(\rho) = Eu(\rho).
\end{equation*}
%
Multiplying the above equation by $2m\alpha^2 / \hbar^2$
%
\begin{equation*}
    -\frac{d^2u(\rho)}{d\rho^2} + \frac{mk}{\hbar^2}\alpha^2 \rho^2u(\rho) = \frac{2m\alpha^2}{\hbar^2}
\end{equation*}

\section{Results}
\label{sec:results}

\subsection{Jacobi's method}

\subsection{Quantum dots, single electron}

\subsection{Quantum dots, two electrons}

\section{Conclusions}
\label{sec:conclusions}

%We have managed to reduce a 3 dimentional two body problem to a analytically solvable problem by changing coordinate system from euclidian to radial, and analyizing CM-system(?)




%\begin{figure}[t]
%
%\mbox{\epsfig{figure=filename.eps,width=\linewidth,clip=}}
%
%\caption{Description of figure -- explain all elements, but do not
%draw conclusions here.}
%\label{fig:figure_label}
%\end{figure}



\begin{deluxetable}{lccc}
%\tablewidth{0pt}
\tablecaption{\label{tab:results}}
\tablecomments{Summary of main results.}
\tablecolumns{4}
\tablehead{Column 1  & Column 2 & Column 3 & Column 4}
\startdata
Item 1 & Item 2 & Item 3 & Item 4
\enddata
\end{deluxetable}




\begin{acknowledgements}
\end{acknowledgements}


\begin{thebibliography}{9}

\bibitem{1}
Stig-Nicolai Foyn, Christer Dreierstad
\textit{Project 1}

\end{thebibliography}


\end{document}
