\documentclass{emulateapj}
%\documentclass[12pt,preprint]{aastex}

\usepackage{graphicx}
\usepackage{float}
\usepackage{amsmath}
\usepackage{epsfig,floatflt}



\begin{document}

\title{Project 4}

\author{Christer Dreierstad}

\email{chrisdre@student.matnat.uio.no}

\altaffiltext{1}{Institute of Physics, University of
  Oslo, P.O.\ Box 1029 Blindern, N-0315 Oslo, Norway}


%\date{Received - / Accepted -}

\begin{abstract}

\end{abstract}
\keywords{computational science: Ising model  --- methods: Metropolis, Monte Carlo}

\section{Introduction}
\label{sec:introduction}
To study the phase transitions in a magnetic system we consider the Ising model for a system of finite lattice size. The energy of the system, by the Ising model, is 
%
\begin{gather*}
    E = -J\sum_{< kl >}^N s_k s_l,
\end{gather*}
%
when there is no external magnetic field applied to the system. For solving this system we consider the Metropolis algorithm, which is a Markov chain Monte Carlo method. 

\section{Method}
\label{sec:method}


\section{Results}
\label{sec:results}
\begin{figure}[t]

c)
\begin{figure}
\mbox{\epsfig{figure=EofMCC-GS-T1-L20-1e5.pdf,width=\linewidth,clip=}}
\caption{text}
\label{fig:E(mcc)T1}
\end{figure}

\begin{figure}
\mbox{\epsfig{figure=EofMCC-GS-T1-L20-1e5.pdf,width=\linewidth,clip=}}
\caption{text}
\label{fig:E(mcc)T1}
\end{figure}

\begin{figure}
\mbox{\epsfig{figure=EofMCC-GS-T1-L20-1e5.pdf,width=\linewidth,clip=}}
\caption{text}
\label{fig:E(mcc)T1}
\end{figure}

\begin{figure}
\mbox{\epsfig{figure=EofMCC-GS-T1-L20-1e5.pdf,width=\linewidth,clip=}}
\caption{text}
\label{fig:E(mcc)T1}
\end{figure}

\begin{figure}
\mbox{\epsfig{figure=EofMCC-GS-T1-L20-1e5.pdf,width=\linewidth,clip=}}
\caption{text}
\label{fig:E(mcc)T1}
\end{figure}

\begin{figure}
\mbox{\epsfig{figure=EofMCC-GS-T1-L20-1e5.pdf,width=\linewidth,clip=}}
\caption{text}
\label{fig:E(mcc)T1}
\end{figure}

MofMCC-GS-T1-L20-1e5.pdf

EofMCC-GS-T2_4-L20-1e5.pdf
MofMCC-GS-T2_4-L20-1e5.pdf

EofMCC-T1-L20-1e6.pdf
MofMCC-T1-L20-1e6.pdf

EofMCC-T2_4-L20-1e6.pdf
MofMCC-T2_4-L20-1e6.pdf



accepts_of_T-T1T2_4-L20.pdf
accepts-T1T2_4-L20-1e5.pdf
CV-T2-23.pdf
CV-T22-24.pdf
E-T2-23.pdf
E-T22-24.pdf
FittedTc.pdf
M-T2-23.pdf
M-T22-24.pdf
PE-T1-L20-1e6.pdf
PE-T2_4-L20-1e6.pdf
X-T2-23.pdf
X-T22-24.pdf

\begin{figure}
\mbox{\epsfig{figure=,width=\linewidth,clip=}}
\caption{text}
\label{fig:figure_label}
\end{figure}




\section{Discussion}
\label{sec:discussion}




\section{Conclusions}
\label{sec:conclusions}




%\begin{figure}[t]
%
%\mbox{\epsfig{figure=filename.eps,width=\linewidth,clip=}}
%
%\caption{Description of figure -- explain all elements, but do not
%draw conclusions here.}
%\label{fig:figure_label}
%\end{figure}



%\begin{deluxetable}{lccc}
%\tablewidth{0pt}
%\tablecaption{\label{tab:results}}
%\tablecomments{Summary of main results.}
%\tablecolumns{4}
%\tablehead{Column 1  & Column 2 & Column 3 & Column 4}
%\startdata
%Item 1 & Item 2 & Item 3 & Item 4
%\enddata
%\end{deluxetable}



\begin{acknowledgements}

\end{acknowledgements}

\begin{thebibliography}{}

\bibitem[G{\'o}rski et al.(1994)]{gorski:1994} G{\'o}rski, K. M.,
  Hinshaw, G., Banday, A. J., Bennett, C. L., Wright, E. L., Kogut,
  A., Smoot, G. F., and Lubin, P.\ 1994, ApJL, 430, 89

\end{thebibliography}


\end{document}
