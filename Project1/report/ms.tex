\documentclass{emulateapj}
%\documentclass[12pt,preprint]{aastex}

\usepackage{graphicx}
\usepackage{float}
\usepackage{amsmath}
\usepackage{epsfig,floatflt}



\begin{document}

\title{A spectacular title}

\author{Ola D. Nordmann}

\email{ola.d.nordmann@astro.uio.no}

\altaffiltext{1}{Institute of Theoretical Astrophysics, University of
  Oslo, P.O.\ Box 1029 Blindern, N-0315 Oslo, Norway}


%\date{Received - / Accepted -}

\begin{abstract}
  State problem. Briefly describe method and data. Summarize main results.
\end{abstract}
\keywords{cosmic microwave background --- cosmology: observations --- methods: statistical}

\section{Introduction}
\label{sec:introduction}

Discuss background, physical importance and possibly some history of
the problem that is being studied in this paper.


\section{Method}
\label{sec:method}


Describe method. Define data model and likelihood. Outline how the
likelihood was computed (grid or MCMC).

Define the power law model in terms of $Q$ and $n$. 

\section{Data}
\label{sec:data}

Summarize properties of data. Which data are used (experiment,
frequencies etc.)? Pixel resolution ($N_{\textrm{side}}$),
$\ell_{\textrm{max}}$ -- everything necessary to repeat the analysis
for other researchers.

Show a sky map of the smoothed data. Use the Healpix routine
``smoothing'' to do this; it works just like anafast. Smooth with a
$7^{\circ}$ beam, and plot with ``map2gif''. Show the RMS pattern as
well. 

\section{Results}
\label{sec:results}


Show the 2D likelihood contours. Summarize constraints on $Q$ and
$n$. 


\section{Conclusions}
\label{sec:conclusions}

Summarize results. Discuss their importance, referring to the
discovery to the initial seeds for structure formation. Mention that
these results are in good agreement with expectations from
inflationary theory.



%\begin{figure}[t]
%
%\mbox{\epsfig{figure=filename.eps,width=\linewidth,clip=}}
%
%\caption{Description of figure -- explain all elements, but do not
%draw conclusions here.}
%\label{fig:figure_label}
%\end{figure}



\begin{deluxetable}{lccc}
%\tablewidth{0pt}
\tablecaption{\label{tab:results}}
\tablecomments{Summary of main results.}
\tablecolumns{4}
\tablehead{Column 1  & Column 2 & Column 3 & Column 4}
\startdata
Item 1 & Item 2 & Item 3 & Item 4
\enddata
\end{deluxetable}



\begin{acknowledgements}
  Who do you want to thank for helping out with this project?
\end{acknowledgements}

\begin{thebibliography}{}

\bibitem[G{\'o}rski et al.(1994)]{gorski:1994} G{\'o}rski, K. M.,
  Hinshaw, G., Banday, A. J., Bennett, C. L., Wright, E. L., Kogut,
  A., Smoot, G. F., and Lubin, P.\ 1994, ApJL, 430, 89

\end{thebibliography}


\end{document}
